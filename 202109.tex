\documentclass[b5j, fleqn, twocolumn]{jsarticle}
\title{計算テスト解説 第6回}
\usepackage{amsmath}
\usepackage{euler}
\usepackage{umoline}
\usepackage{ulem}
\usepackage[top=10truemm,bottom=10truemm,left=10truemm,right=10truemm]{geometry}

\parindent = 0pt
\pagestyle{empty}

\begin{document}
\part*{計算テスト解説 第6回}

\section*{\textcircled{\small{\textrm{7}}}}
\[
  \frac{7}{3},\quad \frac{3}{2},\quad \frac{7}{4},\quad
  \frac{2}{6},\quad \frac{21}{12},\quad \frac{24}{51},\quad
  1.75,\quad 1\frac{8}{16}
\]
分数同士が等しいかを調べるには、すべて既約分数(これ以上約分できない分数)にすれば分かる。
帯分数は仮分数に直す。すべて書き直すとこうなる。
\[
  \frac{7}{3},\quad \uuline{\frac{3}{2}},\quad \uline{\frac{7}{4}},\quad
  \frac{1}{3},\quad \uline{\frac{7}{4}},\quad \frac{8}{17},\quad
  \uline{\frac{7}{4}},\quad \uuline{\frac{3}{2}}
\]

$\dfrac{3}{2}$、$\dfrac{7}{4}$が$2$回以上出てくるから、それぞれが等しい数字である。
\[
  \uline{答.\quad \frac{3}{2}、1\frac{8}{16}は等しい。また、\frac{7}{4}、\frac{21}{12}、1.75は等しい。}
\]

\section*{\textcircled{\small{\textrm{8}}}}
$3$割増しとは$30\%$増しということなので、仕入れ値(原価)を$100\%$として定価は$130\%$ということになる。
\begin{align*}
  & \fbox{  }_{円} \times 1.3 = 845_{円} \\
  \rightarrow \quad & 845_{円} \div 1.3 = \fbox{  }_{円} \\
  \rightarrow \quad & \fbox{  }_{円} = 650_{円}
\end{align*}
\[\uline{答. \quad
  650円
}\]

\section*{\textcircled{\small{\textrm{9}}}}
時速$42\mathrm{km}$で$1時間30分$かかる道のりは、
\begin{align*}
  42\mathrm{km/時} \times \frac{3}{2}時間 &= 63\mathrm{km} \\
  & = 63000 \mathrm{m}
\end{align*}
これを分速$200 \mathrm{m}$で進むと、
\begin{align*}
  63000 \mathrm{m} \div 200 \mathrm{m/分} &= 315 \mathrm{分} \\
  &= 5時間15分
\end{align*}
\[\uline{答. \quad
  5時間15分
}\]

\section*{\textcircled{\small{\textrm{10}}}}
$1$分間で$15L$ずつ入れると$40$分でいっぱいになる水そうなので、水が入る量は
\begin{align*}
  15 \mathrm{L/分} \times 40 \mathrm{分} &= 600 \mathrm{L}
\end{align*}
$600L$の水そうを$30分$でいっぱいにしたいから、
\begin{align*}
  600\mathrm{L} \div 30\mathrm{分} = 20 \mathrm{L/分}
\end{align*}
となって、$1$分間に$20L$ずつ水を入れればいいことが分かる。
\[\uline{答. \quad
  20L
}\]

\end{document}
