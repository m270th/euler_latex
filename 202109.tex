\documentclass[b5j, fleqn]{jsarticle}
\title{計算テスト解説}
\author{算数教室オイラー}
\usepackage{amsmath}
\usepackage{euler}
\usepackage{umoline}
\usepackage{ulem}

\begin{document}
\section{第6回}
  \subsection{10}
  $5$人の平均は、$5$人の点を合計して人数で割ることで求められる。逆に、平均に人数をかけると合計点が分かる。
  今回の$5$人の平均点は$70.8$点なので、$70.8\times5=354$、$5$人の合計点は$354$点となる。\\
  さらに、$A$,$B$ 2人の平均は$78$点なので、$A$,$B$ $2$人の合計点は$78\times2=159点$である。ということは、

  \subsection{11}
  \[
    \frac{7}{3},\quad \frac{3}{2},\quad \frac{7}{4},\quad
    \frac{2}{6},\quad \frac{21}{12},\quad \frac{24}{51},\quad
    1.75,\quad 1\frac{8}{16}
  \]
  分数同士が等しいかを調べるには、すべて既約分数(これ以上約分できない分数)にすれば分かる。
  帯分数は仮分数に変える。書き直すとこうなる。
  \[
    \frac{7}{3},\quad \uuline{\frac{3}{2}},\quad \uline{\frac{7}{4}},\quad
    \frac{1}{3},\quad \uline{\frac{7}{4}},\quad \frac{8}{17},\quad
    \uline{\frac{7}{4}},\quad \uuline{\frac{3}{2}}
  \]

  $\dfrac{3}{2}$、$\dfrac{7}{4}$が$2$回以上出てくるから、それぞれが等しい数字である。
  \[
    \uline{答.\quad \frac{3}{2}、1\frac{8}{16}は等しい。また、\frac{7}{4}、\frac{21}{12}、1.75は等しい。}
  \]

  \subsection{12}
  $3$割増しの定価ということは$30\%$増しということなので、仕入れ値(原価)を$100\%$として定価は$130\%$ということになる。
  \begin{align*}
    & \fbox{  }_{円} \times 1.3 = 845_{円} \\
    \rightarrow \quad & 845_{円} \div 1.3 = \fbox{  }_{円} \\
    \rightarrow \quad & \fbox{  }_{円} = 650_{円}
  \end{align*}
  \[\uline{答. \quad
    650円
  }\]

  \subsection{13}
  時速$42\mathrm{km}$で$1時間30分$かかる道のりは、
  \begin{align*}
    42\mathrm{km/時} \times \frac{3}{2}時間 = 63\mathrm{km}
  \end{align*}

\end{document}